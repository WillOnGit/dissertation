% The structure of this proof is fairly convoluted. It can be broken down as follows:
% Introduce homomorphism
% Easy case
% Difficult case
% 	Establish R, S and T
% 	Make initial argument about Mp
% 	Manipulate [RRR]
% 	Introduce Gauss sums
% 	Introduce f(t)
% 	Calculate f(t)
% 	Calculate discriminant (???)
% 	Argue about g(t)
% 	Prove A satisfies the useful conditions

\begin{theorem}
Define $M_p$ to be the number of projective solutions to the equation $x^3 + y^3 + z^3 = 0$ in the field $\fp$. Then,
\begin{itemize}
\item if $p \not\equiv 1 \mod{3}$, then $M_p = p + 1$.
\item otherwise $p \equiv 1 \mod{3}$, and there exist integers $A$ and $B$ such that
	$$4p = A^2 + 27B^2$$
	Since $A^2 \equiv 1 \mod{3}$, $A \equiv \pm1 \mod{3}$. If we fix the sign of A so that $A \equiv 1 \mod{3}$, then
	$$M_p = p + 1 + A$$
\end{itemize}
\end{theorem}

\begin{proof}
The proof of the two separate cases both rely on the same idea, but in the second case requires many more steps before the result is reached, and beginning the proof without an idea of what is to come can make it very hard to understand, as numerous objects are introduced. Therefore we will prove the first case, provide an outline of the proof second case and then finally supply the details.

For both cases, the proof revolves around examining the homomorphism
$$\varphi: \fpu \rightarrow \fpu \qquad x \mapsto x^3$$
Now in the first case, since 3 does not divide the order of $\fpu$, $\varphi$ is bijective, and so an isomorphism. This means that each element of $\fp$ has exactly one cube root. For example, in $\fp[11]$, we have:
\begin{align*}
	0^3 &= 0,\quad
	1^3 = 1,\quad
	7^3 = 2,\quad
	9^3 = 3,\quad
	5^3 = 4,\quad
	3^3 = 5,\\
	8^3 &= 6,\quad
	6^3 = 7,\quad
	2^3 = 8,\quad
	4^3 = 9,\quad
	10^3 = 10.
\end{align*}
This implies that every solution of $x + y + z = 0$ gives exactly one solution of $x^3 + y^3 + z^3 = 0$ by cube rooting each of $x,y,z$. Since $x + y + z = 0$ is the equation of a line in the projective plane, it has exactly $p+1$ solutions. This settles the first case.

Now to outline what happens in the second case.
\begin{itemize}
\item We start off again by examining the homomorphism $\varphi$, and construct three sets $R$, $S$ and $T$ that together make up $\fp$
\item By introducing a new notation, we then establish an initial expression for $M_p$
\item Manipulate this expression
\item Introduce two new objects, ``cubic Gauss sums'' and an as yet unknown polynomial, $f(t)$
\item Calculate $f(t)$ and its discriminant
\item Use this to update our expression for $M_p$ to its final form
\item Prove that the expression has the necessary properties
\end{itemize}
So after getting an initial expression for $M_p$, we basically update it a lot through many smaller arguments, eventually arriving at the one we need.

Since $p \equiv 1 \mod{3}$, 3 divides the order of $\fpu = p-1$, and so $\varphi$ is a homomorphism that is neither injective nor surjective. For example, in $\fp[7]$, we have:
\begin{align*}
	0^3 &= 0\\
	1^3 = 3^3 = 9^3 &= 1\\
	7^3 = 8^3 = 11^3 &= 5\\
	2^3 = 5^3 = 6^3 &= 8\\
	4^3 = 10^3 = 12^3 &= 12\\
\end{align*}
Now, we can introduce our sets $R$, $S$ and $T$.
\end{proof}
