\begin{definition}
	The modular exponentiation algorithm allows highly efficient calculation of expressions of the form $a^k$ mod $n$, even when the numbers involved are very large. To improve on the naive method of repeatedly multiplying by $a$ and reducing mod n, observe that, for $k=500$,
$$500=2^2+2^4+2^5+2^6+2^7+2^8$$
and therefore $a^k$ may be written as
$$a^{500} = a^{2^{2}} \times a^{2^{4}} \times a^{2^{5}} \times a^{2^{6}} \times a^{2^{7}} \times a^{2^{8}}$$
these terms are much easier to calculate by successively squaring and reducing mod $n$. Let $A_0 = a$ and calculate
\begin{align*}
	A_1\equiv A_0\cdot &A_0 \equiv a^2 \mod n\\
	A_2\equiv A_1\cdot &A_1 \equiv a^4 \mod n\\
	A_3\equiv A_2\cdot &A_2 \equiv a^8 \mod n\\
	&\vdots\\
	A_l\equiv A_{l-1}\cdot &A_{l-1} \equiv a^{2^{l}} \mod n
\end{align*}
and then calculate $a^{500} = A_2 \times A_4 \times A_5 \times A_6 \times A_7 \times A_8 \mod n$.
\end{definition}
As the numbers involved grow larger, the time savings over the trivial method become enormous, as this is an $O(log_2(k))$ algorithm, and it is this algorithm which we shall use in our arithmetic.
