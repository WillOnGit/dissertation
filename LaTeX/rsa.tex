% This .tex file is the example rsa decryption. May need to modify this slightly depending on whether rsa has been introduced at this stage or not.

Now that Lenstra's algorithm has been introduced, we can see a slightly contrived example of a typical use case. Remembering how the algorithm works, say we intercepted a message $M$ along with the public key $P = (e,N)$. If we were able to factor $N$, we would be able to deduce the decryption key, and thus read this and any future messages. 

Let's see how this would work with an example. Say we knew that
$$ M = 3103229009940552729864,\quad P = (65537, 3160853182090460427047) $$
and we wanted to read $M$. If we apply Lenstra's algorithm to $N = 3160853182090460427047$, eventually, taking $(3,1)$ to be our initial point, the algorithm finds that with $b = 39850$, adding the two points
$$(2191801374392476491053, 2332211434379395076998)\ \text{and}\ (406058948051076877967, 3156968592727602662096)$$
is impossible, since 
$$2191801374392476491053-406058948051076877967 = 1785742426341399613086$$ and $\gcd(1785742426341399613086, 3160853182090460427047) = 3992747141$. Of course, we don't mind that we can't add the two points, since $3992747141$ is a non-trivial factor of our modulus. A simple division then gives $N = 3992747141 \times 791648724667$. Because we know this is an RSA modulus, we know that this is the complete factorisation of $N$, but for completeness, a simple brute force check quickly shows that $p = 791648724667$ and $q = 3992747141$ are both prime.

Now that we know this, we calculate $(p-1) \times (q-1) = 3160853181294818955240$ and then find the multiplicative inverse of $e = 65537$ modulo $3160853181294818955240$ with the euclidean algorithm. This turns out to be $146040609624497792033$, and we finally decrypt our message by taking
$$ M^{146040609624497792033} \mod 3160853182090460427047 =2016 $$
