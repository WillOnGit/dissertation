% Don't forget to talk about how you spent ages writing a ======= implementation of the alg!
\subsubsection{The modular exponentiation algorithm}
Before we introduce Lenstra's elliptic curve factorisation algorithm, a prerequisite algorithm, the modular exponentiation algorithm.
This, along with the euclidean algorithm, is one of the two dependencies that Lenstra's algorithm requires.
Therefore, a python implementation of it is also included in \cref{codeappendix}.
\begin{definition}
	The modular exponentiation algorithm allows highly efficient calculation of expressions of the form $a^k$ mod $n$, even when the numbers involved are very large.
	To improve on the naïve method of repeatedly multiplying by $a$ and reducing mod n, observe that, for $k=500$,
$$500=2^2+2^4+2^5+2^6+2^7+2^8$$
and therefore $a^k$ may be written as
$$a^{500} = a^{2^{2}} \times a^{2^{4}} \times a^{2^{5}} \times a^{2^{6}} \times a^{2^{7}} \times a^{2^{8}}$$
these terms are much easier to calculate by successively squaring and reducing mod $n$.
Let $A_0 = a$ and calculate
\begin{align*}
	A_1\equiv A_0\cdot &A_0 \equiv a^2 \mod n\\
	A_2\equiv A_1\cdot &A_1 \equiv a^4 \mod n\\
	A_3\equiv A_2\cdot &A_2 \equiv a^8 \mod n\\
	&\vdots\\
	A_l\equiv A_{l-1}\cdot &A_{l-1} \equiv a^{2^{l}} \mod n
\end{align*}
and then calculate $a^{500} = A_2 \times A_4 \times A_5 \times A_6 \times A_7 \times A_8 \mod n$.
\end{definition}
As the numbers involved grow larger, the time savings over the trivial method become enormous, as this is an $O(\log_2(k))$ algorithm, and it is this algorithm which we shall use in our arithmetic.

\subsubsection{Fermat's little theorem}
Fermat's little theorem is a useful result in number theory that can (sometimes) be used to prove that a number is composite.
\begin{theorem}
	If $p$ is prime, then
	$$2^{p-1} \equiv 1 \mod p.$$
\end{theorem}
Thus if we calculate $2^{n-1} \mod n$ for some $n\in\N$ and find that it does not equal 1, then we know that $n$ is composite
The converse statement is not true, however, and so we see that $2^{n-1} \equiv 1 \mod n$ is a necessary but not sufficient condition for $n$ to be prime.

\subsubsection{Lenstra's algorithm}
Now we are ready to introduce Lenstra's algorithm, which is essentially an improvement on the older $p-1$ algorithm created by John Pollard \cite{tate2013,pollard1974}:
\begin{definition}
	Lenstra's algorithm goes as follows to factor a composite integer $N$:
	\begin{itemize}
		\item Choose random integers $b$, $x$ and $y \mod N$.
		\item Let $P = (x,y)$ and, for a given $b$, let $c = y^2 - x^3 - bx$ such that $P$ is a point on the curve $C: Y^2 = X^3 +bX + c \mod N$.
			In this way, $b$ determines the curve $C$.
		\item Compute $kP$ for large $k$ ($k=10!$, for example).
		\item If the computation of $kP$ is successful, increment $b$ to get a new curve and recalculate $kP$.
		\item Continue until one of the additions fails.
	\end{itemize}
\end{definition}
% What we are doing is taking an integer $N$ that we know is composite and attempting to use it as the modulus in a finite field $\fp[N]$.
% But because it is composite, we know that $\fp[N]$ will not be a proper field, as some elements will not have multiplicative inverses.
% If we attempt to calculate the inverses of these elements with the euclidean algorithm, we will be unable to, as they will not be coprime with $N$.
% But this very calculation will give us a non-trivial divisor of $N$, so it will not matter that we cannot calculate the inverses.
Here, $k$ simply refers to an exponent in $\N$, instead of the field associated with affine and projective space.

When calculating $kP$, we take a similar approach to the modular exponentiation algorithm, in that instead of calculating the $k$-fold sum $\underbrace{P+P+\ldots+P}_{k\text{ summands}}$, we calculate the binary expansion of $k$ and write
$$k = a_0 2^0 + a_1 2^1 + \ldots + a_i 2^i + \ldots + a_n 2^n,$$
where $n\in\N$ and each $a_i$ is either $0$ or $1$.
It follows that
\begin{equation}
kP = a_0 2^0 P + a_1 2^1 P + \ldots + a_i 2^i P + \ldots + a_n 2^n P,
\label{easykp}
\end{equation}
which we can calculate by repeatedly doubling $P$ up to $2^n P$ and adding the points that have nonzero coefficients in \cref{easykp}.

For example, say we want to factor $N = 2638661449034729$.
We can check that $N$ is composite with fermat's little theorem, as mentioned earlier.
Using the modular exponentiation algorithm, we calculate $2^{N-1} \mod N = 1386068300154542$, which proves that $N$ is composite.
Now we choose $x=2$ and $y=1$ such that our point $P = (2,1)$ and let $b = 1$.
Taking $k=10!$, we attempt to calculate $kP$ by doubling $P$ and adding relevant points as described before.
Since $10! = 2^{8} + 2^{9} + 2^{10} + 2^{11} + 2^{12} + 2^{14} + 2^{16} + 2^{17} + 2^{18} + 2^{20} + 2^{21},$ we double $P$ up to $2^{21}P$ and then perform
$$kP = 10!P = 2^{8}P + 2^{9}P + 2^{10}P + 2^{11}P + 2^{12}P + 2^{14}P + 2^{16}P + 2^{17}P + 2^{18}P + 2^{20}P + 2^{21}P.$$
To aid legibility, from now on we write $2^k P = P_k$.

\begin{table}[htbp]
	\centering
	\begin{tabular}{r|l}
		$k$ & $P_k$\\
		\hline
		0 & $(2, 1)$\\
		1 & $(1978996086776085, 1649163405646469)$\\
		2 & $(1354531004190202, 712566094192611)$\\
		3 & $(3034566536053, 2449122987212961)$\\
		4 & $(1301371522167812, 173584894036069)$\\
		5 & $(1996689877431974, 1560461237243599)$\\
		6 & $(1690032118719931, 1115183255675848)$\\
		7 & $(1106989140269887, 1249337210022185)$\\
		8 & $(2006000898376031, 739903289262451)$\\
		9 & $(360409978329708, 680989406049585)$\\
		10 & $(1498346384082037, 1311595726437562)$\\
		11 & $(317917266115269, 1139730538479759)$\\
		12 & $(1176599249124147, 834137694636709)$\\
		13 & $(229831052738857, 33111605954288)$\\
		14 & $(514283736870450, 404971411450613)$\\
		15 & $(1619865029688614, 1758310567012823)$\\
		16 & $(2493326453658282, 878913854327751)$\\
		17 & $(2243262160137209, 1701640423934925)$\\
		18 & $(2323526509553459, 1123682327919074)$\\
		19 & $(777963473649634, 1615413237260867)$\\
		20 & $(1420808458933872, 2034011761100882)$\\
		21 & $(2061892737190217, 240178323260750)$
	\end{tabular}
	\caption{Generating the list of points via doubling when $b=1$}
	\label{unsuccessfuldouble}
\end{table}

To begin with, when $b=1$, we double $P$ as described, up to $P_{21}$.
\Cref{unsuccessfuldouble} shows the points generated by doubling.
Now, we add up the relevant points to find $kP$:
\begin{align*}
	P_{8} &= (2006000898376031, 739903289262451)\\
	+ P_{9} &= (1435983113175084, 866053276340207)\\
	+ P_{10} &= (1372522607116506, 2279199412810534)\\
	+ P_{11} &= (2522377969419304, 1963228719476736)\\
	+ P_{12} &= (1867636445080926, 1800997235517595)\\
	+ P_{14} &= (396572835101474, 143329732066401)\\
	+ P_{16} &= (376321152472975, 2055982586016208)\\
	+ P_{17} &= (1086449006160587, 1492583392899450)\\
	+ P_{18} &= (804870318814085, 1030995341073004)\\
	+ P_{20} &= (1245577332479910, 2193527909976663)\\
	+ P_{21} &= (458028559825619, 2277727531737619)
\end{align*}
Hence, we have successfully calulated
$$kP = (458028559825619,2277727531737619)$$
What now?
Clearly, if we stop here, we have learned nothing about the factorisation of $N$, so we increment $b$ to give a new curve, and start again.
In this particular example, it is possible to calculate $kP$ in this way for $b=2$, $b=3$, \ldots, all the way up to $b=23387$, where something changes.
\Cref{successfuldouble} again shows the list of points generated by doubling $P$, which is successful.
However, when adding points to find $kP$, we find that
\begin{align*}
	P_{8} &= (514795954556322, 1072900592237302)\\
	+ P_{9} &= (1594423138470272, 757463091827773)\\
	+ P_{10} &= (2479979706264660, 2333023450224930)\\
	+ P_{11} &= (2288850814199939, 1125653152307693)\\
	+ P_{12} &= (406974087705183, 2180109173615353)\\
	+ P_{14} &= (2517955941215941, 2377002053335673)\\
	+ P_{16} &= (871996642729558, 279643526813096)\\
	+ P_{17} &= (540192081160502, 2103753450326003)\\
	+ P_{18} &= (1950391654929818, 1926026636369457)\\
	+ P_{20} &= (914230967480337, 1578752046703807)
\end{align*}
This all calculates without issue, as before.
When attempting the final addition
$$(914230967480337, 1578752046703807) + P_{21}$$
we need to calculate the inverse of the difference in $x$-coordinates, as in the addition formulae.
The difference in $x$-coordinates is given by
$$1747536919598074 - 914230967480337 = 833305952117737$$
We then attempt to find the inverse of this with respect to $N$ via the extended euclidean algorithm.
In this instance, though, we see that
\begin{equation}
	\gcd(833305952117737, N) = 33750191
	\label{successfulfactorise}
\end{equation}
and the addition law breaks down.
As a result, from \cref{successfulfactorise} we have found that $33750191$ is a non-trivial factor of $N$, and the algorithm is terminated.
Dividing shows that $N = 33750191 \times 78182119$, and furthermore, these numbers are both prime, so this is the full factorisation of $N$.
\begin{table}[htbp]
	\centering
	\begin{tabular}{r|l}
		$k$ & $P_k$\\
		\hline
		0 & $(2, 1)$\\
		1 & $(1978996223654343, 2307227360301809)$\\
		2 & $(2065341991305728, 2602387696447713)$\\
		3 & $(706090194730739, 364131284052804)$\\
		4 & $(977388443992499, 2229627649103617)$\\
		5 & $(1935153559561631, 2093139597695517)$\\
		6 & $(175553192385993, 2422801076771166)$\\
		7 & $(900832477635742, 1826551381265349)$\\
		8 & $(514795954556322, 1072900592237302)$\\
		9 & $(2067644682928109, 2124286622646949)$\\
		10 & $(2301960260745936, 2560653520820230)$\\
		11 & $(2512378902623323, 2537364249709183)$\\
		12 & $(2441105821836409, 1109262897760599)$\\
		13 & $(1236261064186840, 2217234011447721)$\\
		14 & $(384156417337880, 1320270218117776)$\\
		15 & $(1306192093717332, 688349719246224)$\\
		16 & $(2062019132081890, 832733545528307)$\\
		17 & $(1656673193668733, 916976917613322)$\\
		18 & $(17008362074755, 1666632592566998)$\\
		19 & $(1506916095423602, 1182578326715926)$\\
		20 & $(850331238187406, 2319495258218643)$\\
		21 & $(1747536919598074, 1619293033009998)$
	\end{tabular}
	\caption{Generating the list of points via doubling when $b=23387$}
	\label{successfuldouble}
\end{table}
