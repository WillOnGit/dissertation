% Don't forget to talk about how you spent ages writing a ======= implementation of the alg!
Before we introduce Lenstra's elliptic curve factorisation algorithm, a prerequisite algorithm, the modular exponentiation algorithm.
This, along with the euclidean algorithm, is one of the two dependencies that Lenstra's algorithm requires.
Therefore, a python implementation of it is also included in \Cref{codeappendix}.
\begin{definition}
	The modular exponentiation algorithm allows highly efficient calculation of expressions of the form $a^k$ mod $n$, even when the numbers involved are very large.
	To improve on the naïve method of repeatedly multiplying by $a$ and reducing mod n, observe that, for $k=500$,
$$500=2^2+2^4+2^5+2^6+2^7+2^8$$
and therefore $a^k$ may be written as
$$a^{500} = a^{2^{2}} \times a^{2^{4}} \times a^{2^{5}} \times a^{2^{6}} \times a^{2^{7}} \times a^{2^{8}}$$
these terms are much easier to calculate by successively squaring and reducing mod $n$.
Let $A_0 = a$ and calculate
\begin{align*}
	A_1\equiv A_0\cdot &A_0 \equiv a^2 \mod n\\
	A_2\equiv A_1\cdot &A_1 \equiv a^4 \mod n\\
	A_3\equiv A_2\cdot &A_2 \equiv a^8 \mod n\\
	&\vdots\\
	A_l\equiv A_{l-1}\cdot &A_{l-1} \equiv a^{2^{l}} \mod n
\end{align*}
and then calculate $a^{500} = A_2 \times A_4 \times A_5 \times A_6 \times A_7 \times A_8 \mod n$.
\end{definition}
As the numbers involved grow larger, the time savings over the trivial method become enormous, as this is an $O(\log_2(k))$ algorithm, and it is this algorithm which we shall use in our arithmetic.

%fermat's little theorem?

Now we are ready to introduce Lenstra's algorithm:
\begin{definition}
	Lenstra's algorithm goes as follows to factor an integer $N$ which is known to be composite:
	\begin{itemize}
		\item Choose random integers $b$, $x$ and $y \mod N$
		\item Let $P = (x,y)$ and let $c = y^2 - x^3 - bx$ such that $P$ is a point on the curve $C: Y^2 = X^3 +bX + c \mod N$
		\item Compute $kP$ for large $k$ ($k=10!$, for example)
		\item If the computation of $kP$ is successful, increment $b$ and restart
		\item Continue until one of the additions fails
	\end{itemize}
\end{definition}
% Whilst this may seem strange, considering everything we have discussed so far, it makes sense.
% What we are doing is taking an integer $N$ that we know is composite and attempting to use it as the modulus in a finite field $\fp[N]$.
% But because it is composite, we know that $\fp[N]$ will not be a proper field, as some elements will not have multiplicative inverses.
% If we attempt to calculate the inverses of these elements with the euclidean algorithm, we will be unable to, as they will not be coprime with $N$.
% But this very calculation will give us a non-trivial divisor of $N$, so it will not matter that we cannot calculate the inverses.

Here's an example.
Say we want to factor $N = 2638661449034729$.
We can check that $N$ is composite with fermat's little theorem, as mentioned earlier.
Using the modular exponentiation algorithm, we calculate $2^{N-1} \mod N = 1386068300154542$, which proves that $N$ is composite.
Now we choose our point $P = (2,1)$ and let $b = 1$.
Taking $k=10!$, we attempt to calculate $kP$.
We do this by taking a similar approach to the modular exponentiation algorithm, in that instead of calculating the $k$-fold sum $\underbrace{P+P+\ldots+P}_{k\text{ summands}}$, we calculate the binary expansion of $k$ and write
$$kP = $$

\begin{table}[htbp]
	\centering
	\begin{tabular}{r|l}
		$k$ & $2^k P$\\
		\hline
		1 & $(2, 1)$\\
		2 & $(1978996086776085, 1649163405646469)$\\
		3 & $(1354531004190202, 712566094192611)$\\
		4 & $(3034566536053, 2449122987212961)$\\
		5 & $(1301371522167812, 173584894036069)$\\
		6 & $(1996689877431974, 1560461237243599)$\\
		7 & $(1690032118719931, 1115183255675848)$\\
		8 & $(1106989140269887, 1249337210022185)$\\
		9 & $(2006000898376031, 739903289262451)$\\
		10 & $(360409978329708, 680989406049585)$\\
		11 & $(1498346384082037, 1311595726437562)$\\
		12 & $(317917266115269, 1139730538479759)$\\
		13 & $(1176599249124147, 834137694636709)$\\
		14 & $(229831052738857, 33111605954288)$\\
		15 & $(514283736870450, 404971411450613)$\\
		16 & $(1619865029688614, 1758310567012823)$\\
		17 & $(2493326453658282, 878913854327751)$\\
		18 & $(2243262160137209, 1701640423934925)$\\
		19 & $(2323526509553459, 1123682327919074)$\\
		20 & $(777963473649634, 1615413237260867)$\\
		21 & $(1420808458933872, 2034011761100882)$\\
		22 & $(2061892737190217, 240178323260750)$
	\end{tabular}
	\label{unsuccessfuldouble}
	\caption{Generating the list of points via doubling when $b=1$}
\end{table}

\begin{table}[htbp]
	\centering
	\begin{tabular}{c}
		$(2006000898376031, 739903289262451)$\\
		$(1435983113175084, 866053276340207)$\\
		$(1372522607116506, 2279199412810534)$\\
		$(2522377969419304, 1963228719476736)$\\
		$(1867636445080926, 1800997235517595)$\\
		$(396572835101474, 143329732066401)$\\
		$(376321152472975, 2055982586016208)$\\
		$(1086449006160587, 1492583392899450)$\\
		$(804870318814085, 1030995341073004)$\\
		$(1245577332479910, 2193527909976663)$\\
		$(458028559825619, 2277727531737619)$\\
		$kP = (458028559825619,2277727531737619)$
	\end{tabular}
	\label{unsuccessfuladd}
	\caption{Adding the doubled points to get $kP$ when $b=1$}
\end{table}

\begin{table}[htbp]
	\centering
	\begin{tabular}{r|l}
		$k$ & $2^k P$\\
		\hline
		1 & $(2, 1)$\\
		2 & $(1978996223654343, 2307227360301809)$\\
		3 & $(2065341991305728, 2602387696447713)$\\
		4 & $(706090194730739, 364131284052804)$\\
		5 & $(977388443992499, 2229627649103617)$\\
		6 & $(1935153559561631, 2093139597695517)$\\
		7 & $(175553192385993, 2422801076771166)$\\
		8 & $(900832477635742, 1826551381265349)$\\
		9 & $(514795954556322, 1072900592237302)$\\
		10 & $(2067644682928109, 2124286622646949)$\\
		11 & $(2301960260745936, 2560653520820230)$\\
		12 & $(2512378902623323, 2537364249709183)$\\
		13 & $(2441105821836409, 1109262897760599)$\\
		14 & $(1236261064186840, 2217234011447721)$\\
		15 & $(384156417337880, 1320270218117776)$\\
		16 & $(1306192093717332, 688349719246224)$\\
		17 & $(2062019132081890, 832733545528307)$\\
		18 & $(1656673193668733, 916976917613322)$\\
		19 & $(17008362074755, 1666632592566998)$\\
		20 & $(1506916095423602, 1182578326715926)$\\
		21 & $(850331238187406, 2319495258218643)$\\
		22 & $(1747536919598074, 1619293033009998)$
	\end{tabular}
	\label{successfuldouble}
	\caption{Generating the list of points via doubling when $b=23387$}
\end{table}

\begin{table}[htbp]
	\centering
	\begin{tabular}{c}
		$(514795954556322, 1072900592237302)$\\
		$(1594423138470272, 757463091827773)$\\
		$(2479979706264660, 2333023450224930)$\\
		$(2288850814199939, 1125653152307693)$\\
		$(406974087705183, 2180109173615353)$\\
		$(2517955941215941, 2377002053335673)$\\
		$(871996642729558, 279643526813096)$\\
		$(540192081160502, 2103753450326003)$\\
		$(1950391654929818, 1926026636369457)$\\
		$(914230967480337, 1578752046703807)$\\
		$(914230967480337, 1578752046703807)$\\
		$(1747536919598074, 1619293033009998)$
	\end{tabular}
	\label{successfuladd}
	\caption{Attempting to add the doubled points to get $kP$ when $b=23387$}
\end{table}
% Found 33750191 as a factor on curve with b = 23387
