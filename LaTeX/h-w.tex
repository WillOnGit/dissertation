% This file will give the statement & example of the Hasse-Weil bound by verifying it on x^3 + x + 1 = y^2 over f_7
% Add in intuitive proof and refer

\begin{theorem}
	For an elliptic curve $C: y^2 = f(x)$ over a finite field $\fp$, the number of points on $C = |C(\fp)| = p + 1 + \epsilon$, where $|\epsilon| \leq 2\sqrt{p}$.
\end{theorem}
This is the statement of Hasse's original result, as it pertains to elliptic curves.
The generalisation comes from introducing the genus $g$ and allowing $C$ to be any curve, so that the theorem reads $|C(\fp)| = p + 1 + \epsilon$, where $|\epsilon| \leq 2g\sqrt{p}$.
Since elliptic curves have genus 1, we get Hasse's result from the extended theorem, known as the Hasse-Weil theorem.
\begin{sproof}
	The proof of this theorem is very technical, and so will be omitted.
	Those interested can consult \cite{hasse1936a,hasse1936b,hasse1936c}.
	We give an intuitive, non-rigorous sketch of proof though, and prove a theorem of Gauss from \cite{tate2013} as a supplement.

	Since there are $p$ possibilites for $x$ in $\fp$, if $f(x)$ is a square in $\fp$ for a particular $x$, then this will yield two points on the curve $C$.
	Otherwise, it will yield no points.
	Intuitively, one would not expect there to be a tendency for $f(x)$ to be a square or not to be a square, and so we assume that each $x$ in $\fp$ has a $50\%$ chance of yielding no points and a $50\%$ chance of yielding two points.
	Therefore, one would expect there to be $p$ points from $x \in \fp$ and one more from the point at infinity $\pai$.
	Allowing for a small error term $\epsilon$, we arrive at
	$$|C(\fp)| = p + 1 + \epsilon.$$
\end{sproof}
We can test the Hasse-Weil estimate by calculating the number of points on an elliptic curve $C$ over a particular finite field $\fp$.
For ease of calculation, we will take $\fp = \fp[7]$ and let $C : y^2 = x^3 + x + 1$.
First, we write out the squares in $\fp[7]$:
\begin{align*}
0^2 = 0\\
1^2 = 1\\
2^2 = 4\\
3^2 = 2\\
4^2 = 2\\
5^2 = 4\\
6^2 = 1
\end{align*}
Then we write out $x^3+x+1$ and match to the relevant squares.
\begin{align*}
0^3 + 0 + 1 = 1\\
1^3 + 1 + 1 = 3\\
2^3 + 2 + 1 = 4\\
3^3 + 3 + 1 = 3\\
4^3 + 4 + 1 = 6\\
5^3 + 5 + 1 = 5\\
6^3 + 6 + 1 = 6
\end{align*}
Which leads us to the four points
\begin{align*}
(0,1)\\
(0,6)\\
(2,2)\\
(2,5)
\end{align*}
Including the point at infinity $\pai$, the Hasse-Weil theorem gives $\#C(\fp[7]) = 7 + 1 + \epsilon$, where $|\epsilon| \leq 2\sqrt{7}$, which bounds the number of points on $C$ as $8\pm5$, or $\#C(\fp[7]) \in \{3,4,\ldots,13\}$.
Clearly, 5 agrees with this estimate.
