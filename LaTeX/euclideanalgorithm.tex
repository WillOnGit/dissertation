\begin{definition}
	The euclidean algorithm takes two numbers and returns their \emph{greatest common divisor (gcd)} by repeated division with remainder.
\end{definition}
\begin{align*}
	\gcd(21,15):21 &= 1\times15 + 6\\
	15 &= 2\times6 + 3\\
	6 &= 2\times3 + 0\\
\end{align*}
So $\gcd(21,15)=3$
Using the steps of the algorithm, it is possible to calculate
\begin{align*}
	\gcd(21,15) = 3 &= 1\times15 - 2\times6\\
	&= 15 - 2\times(21 - 1\times15)\\
	&= 15 - 2\times21 + 2\times15\\
	&=3\times15 - 2\times21
\end{align*}
So $3 = 3\times15 - 2\times21$. This is important, because writing the gcd in this fashion allows us to find the multiplicative inverse of an element $a$ of a finite field, by solving a linear congruence $ax\equiv 1 \mod p$.
