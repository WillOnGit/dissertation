\subsubsection{Naïve counting}
If one has taken an elliptic curve $E$ defined by $y^2 = f(x)$ over a finite field $\fp$, a fundamental question to ask is about how many points there are on the curve.
We have already established that there is still a point at infinity on the curve (in Weierstrass form, at $[0,1,0]$), but apart from that, there are a maximum of $p^2$ points.
The naïve approach, of course, is to simply run through all sets of points $(x,y) \in \an[2](\fp)$ and test whether each point satisfies $y^2 = f(x)$, but depending on the size of the field, this may or may not be practical.
Despite this, a basic example of this method of counting is ran through in \cref{hasseweil}.

There are a few approaches possible to take when considering the number of points on an elliptic curve over a finite field, though, and we will cover a number of them.
\subsubsection{The Hasse-Weil estimate}
\label{hasseweil}
The most fundamental theoretical result in finding the order of $E(\fp)$ is the Hasse-Weil theorem, which bounds the size of the group in terms of the prime modulus.
The theorem was initially proved for elliptic curves by Helmut Hasse in \cite{hasse1936a}, \cite{hasse1936b} and \cite{hasse1936c}, and subsequently generalised to curves of higher genus by André Weil in \cite{weil1948}, proved by Pierre Deligne in \cite{deligne1974}.
% This file will give the statement & example of the Hasse-Weil bound by verifying it on x^3 + x + 1 = 0 over f_7
% Is it necessary to go through the method of finding the points??
% When counting points, presumably it is necessary to include the point at infinity?

\begin{theorem}
For an elliptic curve $C$ over a finite field $\fp$, the number of points on $C = \#C(\fp) = p + 1 + \epsilon$, where $|\epsilon| \leq 2\sqrt{p}$
\end{theorem}
\begin{sproof}
	First, a remark; the statement above is actually a more specific statement of the theorem, which is more general. However, the most general definition requires defining the genus of an algebraic variety, which would be an inappropriate amount of detail for this manuscript, and so we stick to this more specific definition.
	In a similar fashion, the proof of this theorem is also very technical, and so will be omitted. Those interested can consult...
\end{sproof}
We can test the Hasse-Weil estimate by calculating the number of points on an elliptic curve $C$ over a particular finite field $\fp$. For ease of calculation, we will take $\fp = \fp[7]$ and let $C : y^2 = x^3 + x + 1$. First, we write out the squares in $\fp[7]$:
\begin{align*}
0^2 = 0\\
1^2 = 1\\
2^2 = 4\\
3^2 = 2\\
4^2 = 2\\
5^2 = 4\\
6^2 = 1
\end{align*}
Then we write out $x^3+x+1$ and match to the relevant squares.
\begin{align*}
0^3 + 0 + 1 = 1\\
1^3 + 1 + 1 = 3\\
2^3 + 2 + 1 = 4\\
3^3 + 3 + 1 = 3\\
4^3 + 4 + 1 = 6\\
5^3 + 5 + 1 = 5\\
6^3 + 6 + 1 = 6
\end{align*}
Which leads us to the four points
\begin{align*}
(0,1)\\
(0,6)\\
(2,2)\\
(2,5)
\end{align*}
Including the point at infinity $\pai$, the Hasse-Weil theorem gives $\#C(\fp[7]) = 7 + 1 + \epsilon$, where $|\epsilon| \leq 2\sqrt{7}$, which bounds the number of points on $C$ as $8\pm5$, or $\#C(\fp[7]) \in \{3,4,\ldots,13\}$. Clearly, 5 % or 4 if you don't count big O?
agrees with this estimate.

\subsubsection{Schoof's algorithm}
The Hasse-Weil estimate gives an important bound on the number of rational points over an elliptic curve over a finite field, but otherwise does not give the exact number.
For this purpose, in 1985, René Schoof published a paper detailing what is known as \emph{Schoof's algorithm}.
This was the first algorithm of its kind to run in polynomial time, as opposed to exponential time.

We do not go into detail on the algorithm, but mention it for its theoretical significance.
Those who are interested can refer to Schoof's paper detailing the algorithm, \cite{schoof1995}.
\subsubsection{Gauss' theorem}
% The structure of this proof is fairly convoluted. It can be broken down as follows:
% Introduce homomorphism
% Easy case
% Difficult case
% 	Establish R, S and T
% 	Make initial argument about Mp
% 	Manipulate [RRR]
% 	Introduce Gauss sums
% 	Introduce f(t)
% 	Calculate f(t)
% 	Calculate discriminant (???)
% 	Argue about g(t)
% 	Prove A satisfies the useful conditions

\begin{theorem}
Define $M_p$ to be the number of projective solutions to the equation $x^3 + y^3 + z^3 = 0$ in the field $\fp$. Then,
\begin{itemize}
\item if $p \not\equiv 1 \mod{3}$, then $M_p = p + 1$.
\item otherwise $p \equiv 1 \mod{3}$, and there exist integers $A$ and $B$ such that
	$$4p = A^2 + 27B^2$$
	Since $A^2 \equiv 1 \mod{3}$, $A \equiv \pm1 \mod{3}$. If we fix the sign of A so that $A \equiv 1 \mod{3}$, then
	$$M_p = p + 1 + A$$
\end{itemize}
\end{theorem}

\begin{proof}
The proof of the two separate cases both rely on the same idea, but in the second case requires many more steps before the result is reached, and beginning the proof without an idea of what is to come can make it very hard to understand, as numerous objects are introduced. Therefore we will prove the first case, provide an outline of the proof second case and then finally supply the details.

For both cases, the proof revolves around examining the homomorphism
$$\varphi: \fp^{\times} \rightarrow \fp^{\times}, \qquad x \mapsto x^3$$
\end{proof}

