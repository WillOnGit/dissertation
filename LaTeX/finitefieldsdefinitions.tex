% Going to include linear congruences in this section too
\begin{definition}
	A field is a commutative, unital ring in which every non-zero element is invertible
\end{definition}
Recall the group $\zn$, the group of integers modulo some natural number $n$.
For a given $n$, we have already seen that we can make $\zn$ a ring by defining the natural multiplication law.
However, the ring $\zn$ is a field if and only if $n$ is prime, and this is denoted by writing $\zn = \fp$ to emphasise both that the modulus is prime and that the ring is a field.
The reason this is true is obvious when considering inverses; if we attempt to make a field out of $\zn[4]$, we find that $2$ has no inverse, since
\begin{align*}
	2 \times 0 &= 0\\
	2 \times 1 &= 2\\
	2 \times 2 &= 0\\
	2 \times 3 &= 2.
\end{align*}
Therefore, $\zn[4]$ fails to meet the criteria for a field.

The question of finding inverses, you should also remember, has an easy, general solution, which involves the euclidean algorithm, which is recapped here.
\begin{definition}
	The euclidean algorithm takes two numbers and returns their \emph{greatest common divisor (gcd)} by repeated division with remainder.
\end{definition}
\begin{align*}
	\gcd(21,15):21 &= 1\times15 + 6\\
	15 &= 2\times6 + 3\\
	6 &= 2\times3 + 0\\
\end{align*}
So $\gcd(21,15)=3$
Using the steps of the algorithm, it is possible to calculate
\begin{align*}
	\gcd(21,15) = 3 &= 1\times15 - 2\times6\\
	&= 15 - 2\times(21 - 1\times15)\\
	&= 15 - 2\times21 + 2\times15\\
	&=3\times15 - 2\times21
\end{align*}
So $3 = 3\times15 - 2\times21$. This is important, because writing the gcd in this fashion allows us to find the multiplicative inverse of an element $a$ of a finite field, by solving a linear congruence $ax\equiv 1 \mod p$.
\begin{definition}
	A linear congruence is an equation of the form
	$$ax \equiv b \mod n$$
\end{definition}
Recall this basic definition from \texttt{MATH111}. Clearly, the nature of linear congruences lends itself to arithmetic within finite fields.
