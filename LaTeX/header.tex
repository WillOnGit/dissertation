\usepackage[hmargin=2.5cm,vmargin=2.5cm]{geometry} %Favourite. From 390 year 2
%\usepackage[textwidth=16cm,textheight=23cm]{geometry} %Nadia's
\usepackage{graphicx}
\usepackage{natbib}
\usepackage{amsmath}
\usepackage{amssymb}
\usepackage{amsfonts}
\usepackage{tikz}
\usepackage{booktabs}
\usepackage{color}
%\usepackage[lofdepth,lotdepth]{subfig} %difference between this and caption/subcaption??
\usepackage{tikz}
\usepackage{cleveref}
\usepackage{lipsum}
\usepackage{todonotes}
\usepackage{microtype}
\usepackage{enumerate}
\usepackage{listings}
\usepackage{caption}
\usepackage{subcaption}

% format for defining new environments:
% \newenvironment{ <<name>> }{ <<starting text/formatting rules>> }{ <<finishing text/formatting rules>> }

%Proof environment
\newenvironment{proof}
 {
   \begin{trivlist} %This new environment uses a list environment to handle spacing etc.
     \item[] {\bf Proof.} %This starts a new list item and inserts the title ``Proof.'' in bold
 }
 {
    \nolinebreak    %this stops a new line being inserted after proof before qed mark
    \hfill          %this expands to the end of the line, pushing the QED mark to the right margin
    \rule{2mm}{2mm} %this is the QED mark (small black square)
   \end{trivlist}
 }
 
 %sketch of proof environment
 \newenvironment{sproof}
 {
   \begin{trivlist}
     \item[] {\bf Sketch of proof.}
 }
 {
    \nolinebreak
    \hfill
    \rule{2mm}{2mm}
   \end{trivlist}
 }
 
 %Definition environment
\newenvironment{definition}
 {
   \begin{trivlist}
     \item[] {\bf Definition.}
 }
 {
   \end{trivlist}
 }
 
 
%Theorem environment (this is a bit different because there is a \newtheorem command)
%Syntax \newtheorem{ <<name>> }{ <<Displayed Title>> }[ <<Where to take numbering from>> ]
\newtheorem{theorem}{Theorem}[subsection] %this defines a new theorem environment ``theorem'' and numbers it within subsections
\newtheorem{lemma}{Lemma}[subsection]     %Lemma environment
\newtheorem{proposition}{Proposition}[subsection]
\newtheorem{corollary}{Corollary}[subsection]

%Two very useful commands
%Command syntax \newcommand{ <<name>> }[ <<number of arguments>> ]{ <<command content>> } #1, #2 etc mark where arguments are inserted

%Differential - This is upright d (\rm means ``roman'' i.e. upright font). The rest of the command handles spacing.
\newcommand{\ud}{\, {\rm d} \kern-.015em }

%This is bold in math mode, and works for letters, numbers and greek (\mathbf doesn't work for greek).
\newcommand{\bm}[1]{\mbox{\protect\boldmath $ #1 $}}



